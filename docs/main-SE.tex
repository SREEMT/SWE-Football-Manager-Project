\documentclass{article}

%%%%% packages to import %%%%%%
\usepackage{graphicx} % Required for inserting images
\usepackage[useregional]{datetime2}
\usepackage[margin=1in]{geometry}
\usepackage{longtable}

%%%%% title information %%%%%%
\title{CS482/495/496 Software Project Proposal: \\ 
KEEPER the soccer manager website for the youth}
\author{Justin Dorsey, Jordan Lim, Alberto Perches, Oselunosen Ehi-Douglas}
\date{\today}

\begin{document}
\maketitle % make title information appear here

\section{Client Information}
By sharing this client information and the rest of this document, you are stating that this client has provided this project as something they want (not something you created and asked if they wanted), and that they are interested in having you complete this project for your capstone.
% Complete the list items about your client
\begin{itemize}
    \item Client name: Eric Cui
    \item Client title: Teen league manager
    \item Client email address: lcui@loyola.edu
    \item Client employer: Loyola University Maryland
    \item How you know the client: Through our Professor
\end{itemize}

\section{Project Description}
% You must complete the following 4 subsections. The instructor will use this information to determine if your project might be feasible or not.

% Comment out the bracketed instructions as you finish subsections
\subsection{Overview}

The project aims to address the need for a way to manage a soccer league for a region, in addition to connecting with soccer fans around the world to stay up to date with their favorite teams. The project addresses these needs by providing various live services that users can access on the web and receive constant updates. The purpose of this project is to provide a soccer web application for the football (soccer) community in which they can use on a daily basis. Our client, Eric Cui, requested a web application that can manage a youth soccer league, as the product will help track youth soccer games for young soccer players and parents.

It is important to note that the project must also adhere to web safety, as the targeted audience consists of children and teens. The project also holds services for coaches, as the app has team databases to manage their players. On the technical side, admins are given permission to modify teams and matches, ensuring the info on the website is updated for viewers. The project overall appeals the soccer community and a tool that can be heavily relied on for years.

\subsection{Key Features}
\begin{itemize}
    \item Live Match and Chat
    \item Posts, including flagged posts
    \item Match Calendar and Bracket
    \item Match Overview
    \item Notification System
    \item Soccer Team Management and Roster Overview
    \item Invite System
    \item Admin, Manager, Adult, Teen, and Guest Roles
\end{itemize}

\subsection{Why this Project is Interesting}
The project was a challenge and a lesson for our team. For most of us, having this as our first software project, it taught us a lot of things from web development to managing a development team. The project was a way for us to figure out how we worked as a team and how to work as a software developer. In the end, this project made us better than we were at the beginning of the semester, preparing us for our capstone projects and for our careers.

\subsection{Areas of CS required}
[What subfields of computer science seem most likely to be relevant to your project? A capstone must involve multiple.]
\begin{itemize}
    \item Web Development
    \item Software engineering 
    \item Cybersecurity 
    \item Database Management
    \item Networking
    \item Human Computer Interaction
\end{itemize}

\subsection{Potential Concerns and Questions}
[Is there any aspect of this project that makes you unsure if it will work, either due to your own interests/background, or that you aren't sure if it fits the requirements? Are there questions you have about this project that you want instructor feedback about?]

%\subsection{Summary of Efforts to Find a Project}
%(Not necessary for 482) [Briefly list out when/how you've discussed with this client, and if you've discussed with other clients who either didn't work out or didn't respond. If you considered a different project and it didn't work out, why didn't it work out?] 

%[Most CS495 projects end here. The sections below are for CS482 and CS496 software projects].

%\subsection{Comparison to Draft}
%[For CS496 only, focus on highlighting the major differences between the draft proposal in CS495 and this one here. If there are no major differences, you can remove this subsection.]

\section{Requirements}

\subsection{Non-Functional Requirements}
%[Non-functional requirements are just as important as functional requirements. Dont forget to specify them.]

\begin{table}[h!]
\centering
\begin{tabular}{c l c p{9cm}}
\hline
\textbf{ID} & \textbf{NFR Title} & \textbf{Category} & \textbf{Description} \\
\hline
NFR1 & Color Scheme & Visual/Design & The color scheme for the app should be a blue \\ \hline
NFR2 & Permission & Access & Invite/login in and out system \\ \hline
NFR3 & Security & Protection & Verification/Accounts \\ \hline
NFR4 & Managing & Updating  & changing and adding things  \\ \hline
NFR5 & Viewing & Security & Being able to use the website \\ \hline
\end{tabular}
\caption{Non-Functional requirements}
\end{table}


\subsection{Functional Requirements (User Stories)}
%[In CS482, all functional requirements are written as User Stories. In CS496, some projects may use a different template to write the requirements. The table below is an example of writing the Stories. Adapt accordingly to different templates or if you want to display more info.]
Table~\ref{tab:1}

%\newpage
%\begin{table}[hbtp]
%\centering
\begin{longtable}[h]{c l c p{10cm}}
\hline
\textbf{ID} & \textbf{Story Title} & \textbf{Points} & \textbf{Description} \\
\hline
A4 & Managing Teens account & 3 & As an Adult, I want to be able to watch and manage my kids/teens accounts, so that I can look and accept any invites while also seeing what they're doing on their accounts \\ \hline
A11 & Live Scores & 5 & As an Admin, I want to be able to update live scores, so that I can keep viewers up to date on live matches 
\\ \hline
A16 & Full Match Control & 3 & As an Admin, I want to be able to check each match in the brackets, so that I can make sure everything is good to start the match, and to allow teams to enter 
\\ \hline
AD1 & Calendar control & 8 & As an Admin, I want to be able to add matches and change/edit the bracket date so that I can update recent changes on the calendar/matches 
\\ \hline
TM5 & Invite System & 3 & As a Team Manager, I want to invite members to a team, so that I can have a soccer team of 11 or more 
\\ \hline
TM6 & Posting Team Info & 5 & As a Team Manager, I want to manage the team I have, so that I can update/change what's going on with the team
\\ \hline
U3 & Login in/sign up screen & 3 & For all users (except Guest), I want to be able to sign up or log in to the site with an account, so that I can save my data and be able to join/work on a team
\\ \hline
U7 & Home screen & 2 & As a user, for all users, I want to see posts and live matches, so that I can look at the current score/results while also interacting with the post 
\\ \hline
U8 & Using Posts & 2 & For all users (with exceptions), I want to be able to interact with posts, so that I can like/dislike and comment on posts while also knowing that teens can't see flagged posts
\\ \hline
U12 & Watching History & 2 & For all users (except Guest), I want to be able to see the history of games I have played in or watched, so that I can have memories of what I have done or seen
\\ \hline
U13 & Profile management & 2 & For all users (except Guest), I want to be able to go into my own profile, so that I can change my info and description about me
\\ \hline
U15 & Full Match display & 2 & For all users, I want to be able to see time, date, place, teams competing, type (normal, playoff, final), so that I can keep with the recent results of a match
\\ \hline
U2 & Inbox screen & 2 & For all users, As a/an For all users (except Guest), I want to be able to see notifications from the website, so that I can see if I get any news and invites from any games, teams, or posts that have come in
\\ \hline
U17 & Settings screen & 3 & As a/an For all users (except Guest), I want to be able to change anything in my profile, so that I can either delete, switch, or change anything for my account
\\ \hline
U18 & Forget password & 1 & As a/an For all users, I want to be able to reset my password, so that I can update my password in case I forget it
\\ \hline
T19 & Parent permission & 1 & As a/an Teen, I want to be able to have a parent or guardian know that you're signing up, so that the website knows if you're allowed to make an account
\\ \hline
U9 & Profile viewing & 3 & As a/an For all users, I want to be able to look at others' profiles, so that I can see who is who and the stats they have
\\ \hline
T10 & Teen verifier & 2 & As a/an Teen, I want to be able view teams and have a badge that tells you that you’re a teen, so that I can interact with people while being safe
\\ \hline
A14 & Playoff generation & 8 & As a/an Admin, I want to be able to generate games, so that I wont to have manual make different playoffs for the users of the website
\\ \hline
U20 & Creating posts & 5 & As a/an For all user (except Guest, I want to be able to make your own post for people to see, so that I can upload anything about anything on the website \\ \hline

\caption{Functional requirements as User Stories.}\label{tab:1}

\end{longtable}
%\end{table}

\section{System Design}

\subsection{Architecture}
We will be using a Model-View-Controller (MVC) architecture since the Rails API is designed to support MVC. Frontend will utilize react, backend is using Redis and MongoDB, while the Rails API handles all the endpoints and logic.\\
Potential Main Modules:
\begin{itemize}
    \item User Management Module
    \item Team Management Module
    \item League Management Module
    \item Analytics Module
    \item Admin Panel Module
    \item Notification Module
\end{itemize}

\subsection{Diagrams}

Figure 1 shows a simplified class diagram for the live score

Figure 2 shows a simplified class diagram for the live match



\begin{figure}[h]
    \centering
    \includegraphics[width=0.45\textwidth]{images/Live score 2.png}
    \caption{Live Score example}
    \label{fig:class-diagram1}
\end{figure}

\begin{figure}[h]
    \centering
    \includegraphics[width=0.45\textwidth]{images/Live Match.png}
    \caption{Live Match example}
    \label{fig:class-diagram2}
\end{figure}

\subsection{Technology}

Table 3 *shows what technologies we are using*. 

\begin{table}[h]
\centering
\begin{tabular}{|l|l|p{8cm}|}
\hline
\textbf{Tech} & \textbf{Type} & \textbf{Usage} \\ \hline
Ruby & Language & Backend \\ \hline
Rails & Framework & Web application framework \\ \hline
Docker & Containerizer & Containerize the app for consistency \\ \hline
Redis & RAM-Based Database & Quick data retrieval and updates; ideal for real-time data \\ \hline
React & UI Framework & Used for UI design and implementation \\ \hline
SQLite3 & Local SQL based Database & Long-term data storage; performance not critical \\ \hline
minitest & Testing tool & Used to test ruby code\\ \hline
SimpleCov & Coverage Report tool & Used to generate coverage report\\ \hline
\end{tabular}
\caption{Technologies used in the project}
\label{tab:technologies}
\end{table}

\subsection{Coding Standards}
\subsubsection{Naming}

\begin{itemize}
    \item \textbf{Global Variables:} All caps and underscores for spaces.  
    Examples: \texttt{VARIABLE}, \texttt{TEST}, \texttt{MIN\_NUM}
    
    \item \textbf{Classes:} Capitalized and CamelCase.  
    Examples: \texttt{Class}, \texttt{ClassGuide}, \texttt{ExTwo}
    
    \item \textbf{Functions:} Lowercase with underscores for spaces. Avoid using numbers.  
    Examples: \texttt{function}, \texttt{function\_two}, \texttt{print\_all\_one}
    
    \item \textbf{Variables:} Same as functions, but numbers do not require underscores.  
    Examples: \texttt{n}, \texttt{x}, \texttt{x2}, \texttt{print\_status}
    
    \item \textbf{Files:} File names should be descriptive and follow variable naming rules unless another convention is necessary.  
    Examples: \texttt{test\_file.txt}, \texttt{grid\_layout.py}, \texttt{App.jsx}
    
    \item \textbf{Folders:} Same as files. Keep names general and simple.  
    Examples: \texttt{folder}, \texttt{folder\_two}, \texttt{src}
\end{itemize}

\subsubsection{Code Formatting}

\begin{itemize}
    \item Include a blank line between each function.
    \item Use consistent and readable loop structures.
    \item Place comments above functions and significant code blocks.
    \item Inline comments should be aligned to the right of the code line they describe.
    \item Maintain consistency throughout the codebase.
\end{itemize}

\subsubsection{Commenting Rules}

\begin{itemize}
    \item Comments must be placed above each function explaining its purpose.
    \item Comments must be placed above each class.
    \item Comments should be detailed and easy to understand.
    \item Inline comments (for specific lines) must appear to the right of that line.
    \item Comments for code blocks should be placed above the block and explain its function.
    \item Create and follow standard templates for class and function comments.
\end{itemize}

\subsubsection{Coding Principles}

\begin{itemize}
    \item Follow the \textbf{DRY (Don't Repeat Yourself)} principle when designing code.
\end{itemize}
\subsubsection{Testing}
    \begin{itemize}
    \item Only allow code with unit tests and with at least 65\% coverage to be committed.
\end{itemize}

\subsection{Data}
Our data will be in MongoDB. Below is the Entity-Relationship list that we will implement to the database:
\begin{itemize}
    \item \textbf{Entity: Admin}
        \begin{itemize}
            \item Attributes:
                \begin{itemize}
                    \item Name
                    \item Username
                    \item Password
                \end{itemize}
            \item Relationships:
                \begin{itemize}
                    \item MODIFY Teams, Profiles, Matches, Posts, and Calendar
                    \item Can RATE Matches and Posts
                    \item Can COMMENT on Posts
                \end{itemize}
        \end{itemize}
    \item \textbf{Entity: Team Manager}
        \begin{itemize}
            \item Attributes:
                \begin{itemize}
                    \item Name
                    \item Username
                    \item Password
                    \item Team
                \end{itemize}
            \item Relationships:
                \begin{itemize}
                    \item INVITE Profiles
                    \item ADD a Team
                    \item VIEW Teams, Profiles, and Calendar
                    \item Can RATE Matches and Posts
                    \item Can COMMENT on Posts
                \end{itemize}
        \end{itemize}
    \item \textbf{Entity: Adult}
        \begin{itemize}
            \item Attributes:
                \begin{itemize}
                    \item Name
                    \item Username
                    \item Password
                    \item Team
                    \item Children
                \end{itemize}
            \item Relationships:
                \begin{itemize}
                    \item ACCEPT INVITES from Team Manager
                    \item ACCEPT INVITES to their Teen from Team Manager
                    \item VIEW Teams, Profiles, and Calendar
                    \item Can RATE Matches and Posts
                    \item Can COMMENT on Posts
                \end{itemize}
        \end{itemize}
    \item \textbf{Entity: Teen}
        \begin{itemize}
            \item Attributes:
                \begin{itemize}
                    \item Name
                    \item Username
                    \item Password
                    \item Team
                    \item Adult Supervisor
                \end{itemize}
            \item Relationships:
                \begin{itemize}
                    \item CHILD OF a Adult
                    \item VIEW Teams, Profiles, and Calendar
                    \item Can RATE Matches and Posts
                    \item Can COMMENT on Posts
                \end{itemize}
        \end{itemize}
    \item \textbf{Entity: Guest}
        \begin{itemize}
            \item Relationships:
                \begin{itemize}
                    \item VIEW Matches, Teams, Profiles, Posts and Calendar
                \end{itemize}
        \end{itemize}
    \item \textbf{Entity: Team}
        \begin{itemize}
            \item Attributes:
                \begin{itemize}
                    \item Players
                    \item Team Name
                    \item Team Logo
                    \item Team Manager
                \end{itemize}
        \end{itemize}
    \item \textbf{Entity: Match}
        \begin{itemize}
            \item Attributes:
                \begin{itemize}
                    \item Score
                    \item Teams
                    \item Rating
                    \item Comments
                    \item History
                \end{itemize}
            \item Relationships:
                \begin{itemize}
                    \item Can VIEW a Team
                \end{itemize}
        \end{itemize}
    \item \textbf{Entity: Calendar}
        \begin{itemize}
            \item Attributes:
                \begin{itemize}
                    \item Matches
                \end{itemize}
            \item Relationships:
                \begin{itemize}
                    \item VIEWS Matches
                \end{itemize}
        \end{itemize}
    \item \textbf{Entity: Profile}
        \begin{itemize}
            \item Attributes:
                \begin{itemize}
                    \item Name
                    \item Team Name
                \end{itemize}
        \end{itemize}
    \item \textbf{Entity: Post}
        \begin{itemize}
            \item Attributes:
                \begin{itemize}
                    \item Comment
                    \item Reactions
                \end{itemize}
        \end{itemize}
\end{itemize}

\subsection{UI Mocks}
\begin{figure}[h]
    \centering
    \includegraphics[width=0.6\textwidth]{images/Loginscreen .png}
    \caption{login screen with the ability to sign up or reset your password (plus being able to be a guest)}
\end{figure}

\begin{figure}[h]
    \centering
    \includegraphics[width=0.6\textwidth]{images/Homescreen.png}
    \caption{Once you log in or join as a guest, you'll start from the home screen with the option bar on the left}
\end{figure}

\begin{figure}[]
    \centering
    \includegraphics[width=0.5\textwidth]{images/Bracket.png}
    \caption{This is where you can view the current tournament matches, with the option to click on each team for more information.}
\end{figure}


\section{Iterations}

\subsection{Iteration Planning}
%[In CS496, you plan all iterations beforehand. In CS482, you update the planning here at each iteration. ] (remove)

\begin{table}[h!]
\centering
\begin{tabular}{c l p{7cm} c}
\hline
\textbf{Iteration} & \textbf{Dates} & \textbf{Stories} & \textbf{Points} \\
\hline
1 & 10/14 - 10/23 & U7 Homepage screen, U3 Login/sign-up screen, U13 My Account screen, U2 Inbox screen & 11 \\ \hline
2 & 10/28 - 11/06 & U17 Settings screen, U18 Forgot Password, T19 Parent permission & 5 \\ \hline
3 & 11/11 - 12/02 & U15 Full Match display, U9 Profile viewing & 5 \\ \hline
\multicolumn{3}{r}{\bf Total: } & 21 \\ \hline
\end{tabular}
\caption{Iteration Planning for Incremental Deliveries}
\end{table}

\subsection{Iteration/Sprint 1}
\subsubsection{Planning}

\begin{itemize}
    \item [$\star$] Justin/Jordan: U7 Home page screen 2pts 
    \item [$\star$] Alberto/Ose: U3 Login screen 1pt, sign up screen 3pts 
    \item [$\star$] All: U13 My Account screen 3pts  
    \item [$\star$] Alberto: U2 Inbox screen 2pts
\end{itemize}

\subsubsection{Work Done}
We did not complete any stories because we did not fully understand how Ruby works and had conflicts with assignments from other classes that week. So far, we have partially completed U3/U7 (Justin/Jordan does U7, and U3 Alberto/Ose does U7).

\subsubsection{Testing Coverage}
\begin{figure}[!h]
    \centering
    \includegraphics[width=0.9\textwidth]{images/thumbnail_image.png}
    \caption{Test coverage from Alberto's end}
\end{figure}

Figure 6: Iteration 1 test coverage is very minimal. We did not have time to implement tests. 

\subsubsection{Retroespective \& Reflection}
What we have learned from the first iteration as a group is that we need to rely on each other more, rather than having one person be the lead on programming. A lot of people were busy with different things to the point where they couldn't work on this project, in which we should have picked up where they left off. Another reason is that this is a new language that most of the group has not done at all, which makes it harder/longer for others pick up how to do it. Hopefully, in this next iteration, we can have 1 fully finished and work on the smaller stories in this one.


\subsection{Iteration/Sprint 2}
\subsubsection{Planning}
Goal:
\begin{itemize}
    \item [$\star$] Justin/Alberto: U17 Settings screen 3pts
    \item [$\star$] Ose: U18 Forgot Password 1pt
    \item [$\star$] Jordan: T19 Parent permission 1pt
    \item [$\star$] Total points: 5pts
\end{itemize}
Stretch Goal:
\begin{itemize}
    \item [$\star$] T10 Teen verifier 2pts
    \item [$\star$] U20 Creating posts 5pts
    \item [$\star$] Total points: 7pts
\end{itemize}
Previous Goal:
\begin{itemize}
    \item [$\star$] U7 Home page screen 2pts
    \item [$\star$] U3 Login screen / sign up screen 3pts
    \item [$\star$] U13 My Account screen 3pts 
    \item [$\star$] U2 Inbox screen 2pts
\end{itemize}


\subsubsection{Work Done}
Alberto got U2 Inbox Screen completed and fully integrated. This feature has real time notifications and a mark as read feature. Logged in users will receive notifications if that notification is tied to them. He also worked on the backend to implement ActionCable and channel subscriptions for any future features. He also refactored the authentication system in the frontend to make authentication easier in the future. Protected routes were also created to add a layer of security. Ose has partially worked on U18 Forgot Password feature but it is not fully functional. The user model was modified to implement role, date of birth, and names. Other goals have not been met as of writing this.

\subsubsection{Testing Coverage}
\begin{figure}[!h]
    \centering
    \includegraphics[width=0.9\textwidth]{images/Screenshot 2025-11-09 224237.png}
    \caption{Test coverage from Alberto's end}
    \label{fig:coverage-it2}
\end{figure}

Figure~\ref{fig:coverage-it2} We implemented tests for almost every file in the backend. The notifications feature was created and tested with 94.44\% coverage. Tests for most of the other files were created. 

\subsubsection{Retroespective \& Reflection}
What we have learned from this is that we need more teamwork and communication. We also need to have everyone complete and work on their tasks in a timely manner. We cant have people work on their tasks last minute in case there is a major issue with the project as a whole. Implementing GitHub actions is something that will be necessary from now on. People need to commit enough, at least once daily and with good documentation. We need to utilize GitHub issues to track and document any bugs or features being implemented correctly. Using PRs is also included in this. Having other team mebers review each other's code is also something we should do.

\subsection{Iteration/Sprint 3}
\subsubsection{Planning}
Goal:
\begin{itemize}
    \item [$\star$] Justin/Alberto: U15 Full Match Display 2pts
    \item [$\star$] Ose/Jordan: U9 Profile Viewing 3pts
\end{itemize}
Stretch Goal:
\begin{itemize}
    \item [$\star$] U8 Using Posts 2pts
    \item [$\star$] U20 Creating posts 5pts
\end{itemize}
Previous Goal:
\begin{itemize}
    \item [$\star$] U7 Home page screen
    \item [$\star$] U3 Login in/Sign up screen
    \item [$\star$] T19 Parent Permission 
    \item [$\star$] U13 Profile Management
    \item [$\star$] U17 Settings Screen
\end{itemize}

\subsubsection{Work Done}
For this Iteration, we got a lot of previous iterations done while having done the current iterations in mine. First, Alberto worked on the functions of the live match while Justin worked on the template (design) on said page. Second, Ose and Jordan was working on displaying the profiles of users once you click on it. For extra things, Justin worked on the homepage, login screen, and settings. Jordan worked on the homepage and sign-up page, Ose worked on the forgot password page, and Alberto worked on more of the backend of things.

\subsubsection{Testing Coverage}
\begin{figure}[h]
    \centering
    \includegraphics[width=0.6\textwidth]{images/iter3testcoverage.png}
    \caption{Test coverage from Justin's end}
\end{figure}

Figure 8: More testing was created and completed to achieve a total of 81.63\% coverage for the backend. Tests were made for the match feature. We also created tests for the password features. 

\subsubsection{Retroespective \& Reflection}
We achieved a lot of things from our backlog this iteration, while also getting most of our current iteration completed. The challenges we had this Iteration were communication and having problems with merging certain branches. For example, Justin had made a lot of changes in his own branch to which was a huge improvement from previous iterations, but the branch he was in was not on GitHub, so it made it harder to bring it over to Alberto's branch to have it reviewed. Another big thing was a bunch of redesigns from different pages, while being able to click to go to them.


\section{Final Remarks}

\subsection{Overall Progress}
%[Have you completed everything? If so, present evidence on how you brought value to your client, and the overall client satisfaction. Otherwise, estimate how much progress you done and how long it would take to finish this project.]
The overall progress of this  project ended up not being halfway completed in terms of our stories we set up and the quality we wanted to put out for our client. For our user stories we managed to get 9/20 started and done (almost half of the user stories have not been touched), and in the terms of the quality of said user stories we got about 532.3 percent out of the 9 stories done (U7 Home-screen: 33.3, U3 Login/sign: 80, U18 Forgot password: 20, U15 Live match: 50, U17 settings: 65, T19 parent permission: 77, U13 profile: 87, U2 inbox: 80, U9 profile viewing: 40). To estimate on how long it would take to finish this project, it would about 4 more iterations with the given work rate of how our work rate has been.


\subsection{Project Reflection}
%[Your personal reflection on the project. What lessons did you learned. What would you have done differently. How can you do better work in future projects? You may write this as a team or per person (or both)]

\textbf{Group reflection:}

\vspace{0.3cm}
\noindent \textit{Justin:} Overall, what I've learned about doing this project for preparations of the senior project next semester would be better management and faster learning of things. What I would have done differently is better communication with my teammates without relying on just text outside of our meetings. How I can do better work in the future would be by learning the ins and outs of using frontend and backend rather than relying on my teammates to do that, while I just manage and design for the team.
In my personal opinion, even though I am the weakest when it comes to programming, and with that weakness coming across something I have not done before, really holding back the progress in the first Iteration, obviously, I got better at it, but it just annoys me that I could not get anything program-related done that first iteration.

\vspace{0.3cm}
\noindent \textit{Jordan:}

\vspace{0.3cm}
\noindent \textit{Ose:}

\vspace{0.3cm}
\noindent \textit{Alberto:} I learned a lot about planning out larger team-based projects. This was my first attempt at a software project of this scale with a team. I enjoyed working with the tech stack we chose, and I learned a lot, especially with the frontend. I also learned how to use GitHub more effectively with issues, issue referencing, workflows, branch rules, and project boards. I learned a lot about planning and its importance, but I felt like it held me back a bit. I learned a lot about team management, writing documentation, effective commit messages, and more. A lot of the skills and lessons I learned revolved around managing larger-scale projects and management skills. 

I could have done better by planning better and managing my time better. A lot of my issues came from improper planning and time management. I would not manage my time efficiently, and it would cost me. I also would not have chosen to use React, considering the amount of learning I had to do. I enjoyed learning it, but I wanted to have a good final product, and it held me back. Choosing a full Rails stack with Hotwire would have made it easier. We would not have had to go through the effort of connecting the Rails API with the React Frontend. More proper pair programming would have been nice, especially if we are learning new technologies. We should have made a larger effort to do this. I feel like we would have had fewer merge conflicts and would have pushed more code per person if we did this.

\section*{Appendix}
[Justin will post the images of each iteration down here]


\end{document}
